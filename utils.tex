%% -- PARA CITAR CAPITULOS -----------------------------------------------------
\newcommand{\chapterName}[1]{Capítulo~\ref{chap:#1}}
\newcommand{\secName}[1]{Sección~\ref{sec:#1}}
\newcommand{\chapterTitle}[2]{\item[\chapterName{#1}: \nameref{chap:#1}] #2}

\newcommand{\secCite}[1]{(ver Sección~\ref{sec:#1})}
\newcommand{\subCite}[1]{(ver Subsección~\ref{sub:#1})}
\newcommand{\subsCite}[1]{(ver Subsubsección~\ref{ssub:#1})}
\newcommand{\parCite}[1]{(ver Parrafo~\ref{par:#1})}

\newcommand{\figCite}[1]{(ver Figura~\ref{fig:#1})}
\newcommand{\figCites}[2]{(ver Figuras~\ref{fig:#1}\textasciitilde\ref{fig:#2})}

\newcommand{\aneCite}[1]{(ver Anexo~\ref{cha:#1})}

\newcommand{\eqCite}[1]{(ver Ecuación~\ref{eq:#1})}

\newcommand{\tableCite}[1]{(ver Cuadro~\ref{tab:#1})}
\newcommand{\tableCiteRange}[2]{(ver Cuadros~\ref{tab:#1}\textasciitilde\ref{tab:#2})}

\newcommand{\listCite}[1]{(ver Listado~\ref{code:#1})}
\newcommand{\listCites}[2]{(ver Listados~\ref{code:#1}\textasciitilde\ref{code:#2})}

%% -- PARA TABLAS -----------------------------------------------------
\newcommand\bitem[1]{\item \textbf{#1}}
\renewcommand\theadfont{\bfseries}

% para alinear las columnas
\newcolumntype{L}[1]{>{\raggedright\let\newline\\\arraybackslash\hspace{0pt}}m{#1}}
\newcolumntype{C}[1]{>{\centering\let\newline\\\arraybackslash\hspace{0pt}}m{#1}}
\newcolumntype{R}[1]{>{\raggedleft\let\newline\\\arraybackslash\hspace{0pt}}m{#1}}

\newcommand{\myTable}[4][\small]{%
  \begin{savenotes}
    \begin{table}[hp]
      \centering
      \rowcolors{1}{}{lightgray}
      {%
        #1
        \input{./tables/#2.tex}
      }
     \caption{#3}
     \label{tab:#4}
    \end{table}
  \end{savenotes}
}


%% -- PARA ECUACIONES -----------------------------------------------------
\newenvironment{referencedEquation}[1]{%
  \begin{equation} \label{eq:#1}
}{%
  \end{equation}
}


%% -- PARA CITAS -----------------------------------------------------

\makeatletter

\newenvironment{myQuote}[2][2em]{%
  \setlength{\@tempdima}{#1}%
  \def\myQuote@author{#2}%
  \parshape 1 \@tempdima \dimexpr\textwidth-2\@tempdima\relax%
  \itshape
}{%
  \par\normalfont\hfill--\ \myQuote@author\hspace*{\@tempdima}\par\bigskip
}

\newenvironment{myQuoteWithAutor}[2]{%
  \begin{myQuote}{#1, \textit{#2}}
}{%
  \end{myQuote}
}

\makeatother


%% -- PARA ORDENAR UN CONJUNTO DE ITEMS -----------------------------------------------------
% Fuente original: https://tex.stackexchange.com/questions/121489/alphabetically-display-the-items-in-itemize

\newcommand{\sortsimpleitem}[2]{%
\DTLnewrow{list}% Create a new entry
\DTLnewdbentry{list}{sortlabel}{#1}% Add entry sortlabel (no optional argument)
\DTLnewdbentry{list}{description}{\textbf{#1}: #2}
}

\newcommand{\sortitem}[3][\relax]{%
\DTLnewrow{list}% Create a new entry
\ifx#1\relax
    \DTLnewdbentry{list}{sortlabel}{#2}% Add entry sortlabel (no optional argument)
    \DTLnewdbentry{list}{description}{\textbf{#2} \cite{#2}: #3}
\else
    \DTLnewdbentry{list}{sortlabel}{#1}% Add entry sortlabel (optional argument)
    \DTLnewdbentry{list}{description}{\textbf{#2} \cite{#1}: #3}
\fi%
}

\newcommand{\sortitemcustombib}[4][\relax]{%
\DTLnewrow{list}% Create a new entry
\ifx#1\relax
    \DTLnewdbentry{list}{sortlabel}{#3}% Add entry sortlabel (no optional argument)
    \DTLnewdbentry{list}{description}{\textbf{#3} \cite{#2}: #4}
\else
    \DTLnewdbentry{list}{sortlabel}{#1}% Add entry sortlabel (optional argument)
    \DTLnewdbentry{list}{description}{\textbf{#3} \cite{#2}: #4}
\fi%
}

\newenvironment{sortedlist}{%
\DTLifdbexists{list}{\DTLcleardb{list}}{\DTLnewdb{list}}% Create new/discard old list
}{%
\DTLsort{sortlabel}{list}% Sort list
\begin{itemize}%
    \DTLforeach*{list}{\theDesc=description}{%
    \item \theDesc}% Print each item
\end{itemize}%
}

%% -- PARA FIGURAS -----------------------------------------------------
\NewDocumentCommand{\myBaseFigure}{O{1.0}mmmm}{%
  \begin{figure}[!h]
  \begin{center}
  \includegraphics[width=#1\textwidth]{./figures/#2}
  \caption[#3]{#4}
  \label{fig:#5}
  \end{center}
  \end{figure}
}

\NewDocumentCommand{\myFigure}{O{1.0}mmm}{%
  \myBaseFigure[#1]{#2}{#3}{#3}{#4}
}

\NewDocumentCommand{\myTOCFigure}{O{1.0}O{\footnotesize}mmmm}{%
\myBaseFigure[#1]{#3}{#4}{#4\\{#2 #5}}{#6}
}

\NewDocumentCommand{\myTOCJoinFigure}{O{1.0}O{\footnotesize}mmmm}{%
\myBaseFigure[#1]{#3}{#4}{#4 #5}{#6}
}

\NewDocumentCommand{\myReferencedFigure}{O{1.0}O{\footnotesize}mmmm}{%
\myBaseFigure[#1]{#3}{#4}{#4\\{#2 Fuente: #5}}{#6}
}

\NewDocumentCommand{\myFigureRotated}{O{1.0}mmm}{%
  \begin{figure}[!h]
  \begin{center}
  \includegraphics[height=#1\textwidth,angle=90]{./figures/#2}
  \caption{#3}
  \label{fig:#4}
  \end{center}
  \end{figure}
}

%% -- PARA TEXTO ENTRE COMILLAS -----------------------------------------------------
\newcommand{\commentMe}[1]{<<#1>>}
\newcommand{\citeMe}[1]{\commentMe{\textit{#1}}}

%% -- PARA LISTADOS -----------------------------------------------------
\definecolor{backcolour}{rgb}{0.95,0.95,0.92}

\newcommand{\myListing}[5][\footnotesize]{%
  \bgroup                    %%<<-- use this
  \inputminted[
    frame=lines,
    framesep=2mm,
    baselinestretch=1.2,
    %bgcolor=backcolour,
    fontsize=#1,
    linenos,
    breaklines
  ]{#2}{./code/#3}
  \captionof{listing}{#4\label{code:#5}}
  \egroup                  %% and this
}

%% -- PARA PODER DIBUJAR EL TICK DE LOS OBJETIVOS -----------------------------------------------------
\def\checkmark{\tikz\fill[scale=0.4](0,.35) -- (.25,0) -- (1,.7) -- (.25,.15) -- cycle;}

%% -- AYUDAS PARA LA HORA DE ESCRIBIR Y PODER VAGUEAR MAS ---------------------------------------------
\newcommand{\contHerr}{Algunas de estas herramientas son:}

\newcommand{\figMockup}[3][0.7]{\myFigure[#1]{mockup-#2.png}{Boceto de la interfaz de \commentMe{#3}}{mockup-#2}}
\newcommand{\figDiaSeq}[2]{\myFigure[0.7]{diagrama-secuencia-#1.png}{Diagrama de secuencia de \commentMe{#2}}{diagrama-secuencia-#1}}

\newcommand{\myAppScreen}[3][0.9]{%
  \myFigure[#1]{manual/#2.png}{#3}{manual-#2}
}
