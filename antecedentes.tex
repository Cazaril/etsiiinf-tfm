\chapter{Marco Histórico}
\label{chap:marco}
\section{Procesamiento del lenguaje natural}

A pesar de que la idea de máquinas que tradujesen automáticamente se remonta al
menos al siglo diecisiete, durante el cual se proponen sistemas teóricos mediante los
cuales relacionar palabras de un lenguaje a otro de forma unívoca, dichas ideas
o propuestas se quedan en meros ejercicios teóricos muy alejados de una
implementación que pudiese llevar a cabo. Habrá que esperar hasta el siglo
veinte, durante el cual veremos una explosión de avances en este área.

La primera propuesta de patente de una máquina que implementaba un diccionario
bilingüe utilizando tarjetas perforadas se propuso a mediados de la década de
1930. Esto está aún lejos de ser un sistema de traducción automática real,
aunque sí que podría considerarse como una de las primeras -si no la primera-
propuesta de elemento automático-mecánico de apoyo a la traducción, primera
piedra necesaria de los cimientos.
También en la misma época se presentó una propuesta de patente más interesante
en la cual, además de un diccionario bilingüe, se dividía el trabajo en fases
repartidas entre traductores humanos nativos -tanto del lenguaje origen como del
lenguaje objetivo- y una fase intermedia de traducción automática.

Aunque estas propuestas tardarían decadas en ser conocidas, resultan una
primera aproximación a un problema que, como veremos en las siguientes páginas,
no resulta en absoluto trivial.

\subsection{Problemática de la traducción}
En general, pocos de los problemas que veremos a continuación han sido resueltos
de forma plenamente satisfactoria si no que se ha llegado a un punto de
equilibrio en cuanto a coste/beneficio.

\begin{itemize}
\item Problema del contexto. En una frase, entender y traducir el contexto
  no siempre resulta inmediato. La poesía, la ironía y, en general, cualquier
  herramienta lingüistica compleja da lugar a la interpretación subjetiva del
  interlocutor lo cual hace que en ocasiones, incluso los propios expertos
  humanos, traduzcan de forma errónea frases dándoles un sentido que antes
  podían no poseer.
\item Peculiaridades idiomáticas. En todas las lenguas existen conceptos o
  términos que, por su alcance o su peculiaridad, resultan de dificil si no
  imposible traducción. Un ejemplo de esto puede ser `Toska' en Ruso o
  `Saudade' en Portugués, ambos siendo términos de difícil traducción
  directa a otra lengua.
\item La traducción múltiple. Entendiendo como la traducción como el
  ejercicio de transformar elementos de un lenguaje en elementos de otro
  lenguaje, es importante remarcar que, si partimos de un método de
  transformación -Un método de traducir- propenso a errores y a pérdidas de
  contenido lingüistico, la traducción de contenido a partir de otra
  traducción previa y no a partir del texto original hace que la acumulación
  de errores resulte en un efecto parecido al que se podría obtener mediante
  el juego `El telefono estropeado'.
\item Bancos de términos. En cualquier labor de traducción automática, es de
  máxima importancia tener bases de datos donde estén almacenados con el máximo
  nivel de detalle los diferentes términos del idioma; Sin este requisito, es
  inviable abordar cualquier tarea de traducción haciendo la labor de la
  traducción automática intensiva desde un punto de vista computacional o de
  capacidad de memoria.
\item Construcción de bancos de términos. Relacionado con el problema anterior
  existe la dificultad de la construcción de dichos bancos y, aún más
  importante, su mantenimiento y actualización: No podemos olvidar que una
  lengua natural es una herramienta viva, flexible y con significados subjetivos
  muy localizados geográficamente haciendo el mantenimiento de ese banco aún más
  complejo si cabe.
\end{itemize}
Toda esta problemática se traduce en un hecho concreto:
No existe la traducción automática perfecta; Mejor dicho, para textos cuya complejidad
sobrepase a la de un texto de índole infantil, no existe un método sistemático que garantice
una traducción automática sin pérdida de información.

Diversos métodos se han inventado para evitar estos problemas así como otros que
surgieron a raiz de dichos métodos, pero el problema de la traducción automática
-En todas sus implicaciones: Minería de textos, Recuperación de información,
Extracción de la información, etc \ldots- sigue siendo uno de los grandes
problemas del siglo y, en opinión del autor, el problema que más duro de
resolver resultará a lo largo del siglo XXI.

\section{Aprendizaje y Razonamiento artificial}

Este es un campo con multitud de derivaciones y subáreas, cada una con sus
particularidades y desarrollos independientes.

La idea de crear objetos con raciocinio propio y los autómatas ya
formaban parte del imaginario común desde la antiguedad, pero será necesario
esperar hasta el siglo XX para encontrar ejemplos concretos y específicos sólo
atribuibles al campo de la Inteligencia Artificial.

Antes de poder comentar las distintas problemáticas de estas areas, es necesario
dar unas pinceladas de contexto y comentar ciertos conceptos concretos del
presente trabajo.
Tradicionalmente, la Inteligencia Artificial ha buscado la
manera de replicar el mundo natural construyendo modelos artificiales que fuesen
indistinguibles -interior y exteriormente- a los originales: Asumimos que el ser
humano es el ser vivo más inteligente y por tanto se busca replicar en la medida
de lo posible aquello que lo distingue del resto de seres vivos -su cerebro- de
la misma manera de asumimos el modelo de cámara estenopeica como una versión
primitiva y limitada del ojo humano, pero similar en su funcionamiento
fundamental. Esta búsqueca constante de una mejor aproximación ha mantenido
siempre vivo el campo y lo ha hecho ser, en general, un área multidisciplinar
pero también puede haber limitado el desarrollo, obligándolo a continuar una
línea de investigación sobreexplotada.

Siguiendo la analogía de considerar al ser humano como el modelo de inteligencia
a reproducir, consideremos el problema del ajedrez: si entendemos un estado como
la distribución específica de las piezas a lo largo del tablero en un momento
concreto de la partida, es imposible que un ser humano memorice todos los
estados posibles de una partida para, en función del estado actual, encontrar el
estado que mejore su situación lo máximo posible. ¿Cómo resolvemos entonces ese
tipo de problemas los humanos? 

\subsection{Inteligencia artificial}
Siempre hemos entendido la IA como el trabajo a través del cual dotar a máquinas
o métodos automáticos de la capacidad de tomar decisiones o actuar de formas que
habitualmente se atribuyen a la inteligencia.
Pero ¿Qué podemos entender hoy en día cuando decimos `Inteligencia Artificial`?
¿Seguimos teniendo la misma intuición de inteligencia que teníamos hace 30 años?
Históricamente, la IA ha ido de la mano del desarrollo de `Heurísticas', es
decir, reglas más o menos ad-hoc que tienen la particularidad de acertar mucho y
fallar poco. Dichas heurísticas son fruto del descubrimiento o desarrollo de un
agente humano, que previamente ha obtenido esa regla o conjunto de ellas a
través de la experiencia; Una heurística, al final, es una guía fruto de la
experiencia que apunta en la dirección de una posible solución basada en un
conocimiento obtenido ad-hoc.

Obviamente, no todos los problemas admiten de la misma forma una heurística ni
existe una única heurística que permita resolver de forma satisfactoria todos
los problemas, de la misma forma que no existe una única forma de actuar que
nos permita enfrentarnos a todas las situaciones. Pero esta aproximación tiene
la problemática del coste y de necesitar a uno o más expertos en el problema con
la suficiente experiencia como para haber interiorizado una forma de ataque;
El estudio matemático del problema se utiliza para dotar de coherencia
los resultados así como resolver los casos en los que la heurística falle,
proporcionando reglas formales que permitan atacar al problema con una cierta
garantía de seguridad.

Si estudiáramos la evolución de este área, dividir el desarrollo de la IA en 3
grandes olas -actualmente estamos en la cresta de la tercera- con sus
respectivas caídas de interés y popularidad. A lo largo de dichas olas se ha
producido un cambio en la popularidad de ciertas tecnologías, aumentando
aquellas que previamente habían sido rechazadas o no gozaban de popularidad y
disminuyendo aquellas para las cuales se habían hecho demasiadas promesas.



Orígenes de las redes de neuronas.

Orígenes del Big Data. Qué es?

Qué entendemos por razonamiento inteligente.

Problemática concreta del área.
% Este capítulo incluye unas nociones básicas de \LaTeX{} y algunos consejos
% sencillos de composición para sacar todo el jugo a la clase \packageName. Ten
% presente que este capítulo está pensado para que leas el código fuente y lo
% compares con el resultado en PDF

% \section{Estilos de texto}

% Debido a su continuo uso, se muestra entre paréntesis la combinación del modo
% \texttt{auctex} de GNU Emacs para incluir el comando \LaTeX{} correspondiente.

% \begin{itemize}
% \item Normal.
% \item \textbf{Negrita} (C-c-f-b).
% \item \textit{Itálica} (C-c-f-i).
% \item \emph{Enfatizada} (C-c-f-e). Fíjate que el estilo que se obtiene al
%   enfatizar depende del estilo del texto en el que se incluya: \textit{texto en
%     itálica y \emph{enfatizado} en medio}.
% \item \texttt{Monoespaciada} (C-c-f-t)
% \end{itemize}

% Otros de menos uso:

% \begin{itemize}
% \item \textsc{Versalita} (C-c-f-c).
% \item \textsf{Serifa}, es decir, sin remates o paloseco (C-c-f-f).
% \item \textrm{Romana} (C-c-f-r).
% \end{itemize}


% \section{Viñetas y enumerados}

% En \LaTeX{} hay tres tipos básicos de viñetas:

% \begin{itemize}
% \item itemize.
% \item enumerate.
% \item description.
% \end{itemize}


% Es posible hacer viñetas (como la siguiente) cambiando márgenes u otras
% propiedades gracias al paquete
% \href{http://mirror.ctan.org/macros/latex/contrib/enumitem/enumitem.pdf}{\emph{enumitem}}
% (ya incluido en \packageName).

% \begin{itemize}[noitemsep, label=$\triangleright$]
% \item esto es
% \item una pequeña
% \item muestra
% \end{itemize}

% El paquete \emph{enumitem} ofrece muchas otras posibilidades para personalizar
% las viñetas (individual o globalmente) o crear nuevas.

% \section{Cuadros}
% \label{sec:uncuadro}

% Se denominan «tablas» cuando contienen datos con relaciones numéricas. El
% término genérico (que debe usarse cuando en los demás casos) es
% «cuadro»~\cite{sousa}. Si las columnas están bien alineadas, las líneas
% verticales estorban más que ayudan (no las pongas). Los cuadros se referencian
% de este modo \tableCite{rpc-semantics}.

% \begin{table}[hp]
%   \centering
%   {\small
%   \input{tables/RPC-semantics.tex}
%   }
%   \caption[Semánticas de \acs{RPC} en presencia de distintos fallos]
%   {Semánticas de \acs{RPC} en presencia de distintos fallos
%     (\textsc{Puder}~\cite{puder05:_distr_system_archit})}
%   \label{tab:rpc-semantics}
% \end{table}


% \section{Listados de código}
% \label{sec:listado}

% Puedes referenciar un listado así \listCite{hello}. Éste es un
% listado flotante, pero también pueden ser «no flotantes» quitando el parámetro
% \texttt{float} (mira el fuente de este documento o la referencia del paquete
% \href{http://www.ctan.org/get/macros/latex/contrib/listings/listings.pdf}{«listings»}).

% \myListing{C}{hola_mundo.c}{\commentMe{Hola mundo} en C}{hello}

% \noindent
% Y también existe un comando \texttt{console} para representar ejecución de
% comandos:

% \begin{console}
% $ uname --operating-system
% GNU/Linux
% \end{console} %$

% En cualquier caso, si lo necesitas siempre es mejor que redefinas los comandos y entornos
% existentes o crees entornos nuevos, en lugar de añadir los mismos cambios en
% muchas partes del documento.

% \section{Notas}
% \label{sec:notas}

% Es posible añadir notas en el pdf, para ello hay tres:

% \duda{No se hacer XXX}\\
% \nota{Deberia hacer esto}\\
% \arreglar{Cambiar cuanto antes}\\

% Las notas solo se pueden ver si esta activado el modo \commentMe{borrador} en el fichero \commentMe{metadata.tex}.




% \section{Citas y referencias cruzadas}

% Puedes ver aquí una cita~\cite{design_patterns} y una referencia a la segunda sección
% (véase \S\,\ref{sec:uncuadro}). Para hacer referencias debes definir etiquetas en el punto
% que quieras referenciar (normalmente justo debajo). Es útil que los nombres de las
% etiquetas (comando label) tengan los siguientes prefijos (incluyendo los dos puntos ``:''
% del final):

% \begin{description}
% \item[chap:] para los capítulos. Ej: ``\texttt{chap:objetivos}''.
% \item[sec:] para secciones, subsecciones, etc.
% \item[fig:] para las figuras.
% \item[tab:] para las tablas.
% \item[code:] para los listados de código.
% \end{description}

% Si estás viendo la versión PDF de este documento puedes pinchar la cita o el número de
% sección. Son hiper-enlaces que llevan al elemento correspondiente. Todos los elementos que
% hacen referencia a otra cosa (figuras, tablas, listados, secciones, capítulos, citas,
% páginas web, etc.) son «pinchables» gracias al paquete
% \href{http://latex.tugraz.at/_media/docs/hyperref.pdf}{\emph{hyperref}}.

% Para citar páginas web usa el comando \texttt{url} como en: \url{http://www.uclm.es}


% \section{Páginas}
% \label{sec:paginas}

% La normativa aconseja imprimir el documento a doble cara, pero si el número de
% páginas es bajo puede imprimirse a una cara. Como eso es bastante subjetivo, mi
% consejo es que ronde las 100 \textbf{hojas}. Una hoja impresa a doble cara
% contiene 2 páginas, a una cara contiene una. Es decir, si el documento tiene más
% de 200 páginas imprímelo a doble cara, si tiene menos imprímelo a una.

% Por defecto, \packageName{} imprime a una cara (oneside), si quieres imprimir a doble cara,
% escribe en el preámbulo:

% \myListing{latex}{latex_una_pagina.tex}{A una pagina}{una-pagina}


% Esto es importante porque a doble cara los márgenes son simétricos y a una cara
% no. Si llevas el TFM a la copistería y pides que te lo impriman de modo
% diferente al generado, quedará mal ¡Cuidado!

% Tal como indica la normativa, los capítulos siempre empiezan en la página
% derecha, la impar cuando se usa doble cara.


% Local Variables:
%  coding: utf-8
%  mode: latex
%  mode: flyspell
%  ispell-local-dictionary: "castellano8"
% End:
