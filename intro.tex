%&latex2
\chapter{Introducción}
\label{chap:introduccion}

\drop{A} lo largo de las décadas, se han visto las distintas olas de la
Inteligencia Artificial, sus consecuencias y los periodos intermedios conocidos
como inviernos. Existe un punto de similitud entre las distintas olas: Se
prometieron soluciones y avances que nunca llegaron y se menospreció la
dificultad computacional de los problemas. En el contexto actual en el cual se
ha vuelto a hablar de la Inteligencia Artificial, es interesante comprobar
exactamente la calidad de los avances del área y si se ha conseguido un avance
cualitativo con respecto a las técnicas de hace 30 años.

Siempre que se habla de la IA como área, existen dos grandes problemas:
\begin{itemize}
\item El efecto IA.Que resume la dificultad que existe cuando, al conseguir un
  proceso conocido para un problema, dicho proceso pasa a no ser considerado
  IA.Esto es, la misma comprensión de un algoritmo lo despoja de inteligencia
  ya que se elimina ese halo de subjetividad humana que rodea al problema.
\item Como resultado de la explosión de las distintas burbujas en los
  sucesivos inviernos de la IA, se ha establecido una desconfianza hacia todo
  lo relacionado con esta área y muchas de las salidas industriales han
  cambiado de nombre para librarse de ese estigma.
\end{itemize}
  
Estos problemas son solo la punta del iceberg, puesto que al ser un área
multidisciplinar y de investigación, aúna las partes negativas de áreas con
estas características – Necesidad de equipo con muy alta preparación, necesidad
de inversión fuerte, baja tasa de retorno de la inversión, etc – haciéndola un
objetivo de recortes inmediatos en tiempos de escasez económica.

Si podemos dividir el campo de la IA en sus tres pilares históricos fundamentales:
\begin{itemize}
\item Aprendizaje y razonamiento artificial.
\item Robótica.
\item Procesamiento del lenguaje natural.
\end{itemize}

Como es comprensible, la amplitud de un análisis que abarque por completo este
área resulta inviable tanto por extensidad como por alcance del presente
trabajo. Nos centraremos en el mismo en sólo dos de estos pilares: En el
Aprendizaje y razonamiento artificial y en el procesamiento del lenguaje natural
aunque es de rigor comentar la grandísima evolución ocurrida en el campo de la
robótica, aumentando las capacidades de acción y reacción de los distintos
componentes, dotándolos además de mejores sistemas motrices y mejor capacidad
propioceptiva.

Es interesante avanzar el problema de la explosión combinatorial del cual
hablaremos en profundidad más adelante. Los problemas a los que se suele
enfrentar la IA suelen ser problemas cuyo comportamiento responde a esta
explosión: El espacio de soluciones se hace cada vez más grande de forma
exponencial con respecto al tamaño de la muestra.

Este comportamiento hace difícil solucionar los problemas de forma clásica, es
decir, buscando la mejor solución generando todas las posibles soluciones del
problema. Es en este punto en el que entra la IA, proporcionando heurísticas –
más adelante se hablará de ellas – que permitan ahorrarse al algoritmo de
resolución la generación de soluciones no válidas. Dichas heurísticas suelen ser
guías más o menos ingeniosas encontradas por un ser humano pero también pueden
ser el fruto de un algoritmo de categorización automática, que de forma propia
ha creado dichas guías. A pesar de que la IA suele ser la herramienta con la
cual abordar este tipo de problemas, también ha sido necesario el desarrollo
técnico de los ordenadores – potencia de cálculo, cantidad de memoria disponible
– que permitiera a las distintas técnicas medirse con el problema a
resolver. Pero si ha sido necesario ese salto en la capacidad de cómputo, es
aceptable la duda de si dicha técnica es realmente una técnica válida.

Con esta duda en mente, se pretende realizar un análisis exhaustivo de cada área
a tratar, comentando los avances cualitativos o cunatitativos de cada
investigación para después hacer una comparación real con los avances más
recientes. Con esto se pretende hacer, además de un estado del arte, una
revisión crítica de los distintos avances y de su impacto real.

% REVISAR A PARTIR DE AQUÍ REVISAR

% Otro problema interesante y distinto en cuanto a su enfoque inicial es el
% problema de la traducción automática: Obtener traducciones válidas pero
% generadas de forma automática por una máquina. Este problema existe desde hace
% muchísimo tiempo y que conlleva una problemática importante; Desde la traducción
% de conceptos propios de una lengua a la traducción de la semántica propia de la
% oración o el problema de la contextualización de textos.

% El tratamiento y estudio de este tipo de problemas ha requerido una cantidad de
% recursos – generalmente prohibitivos –, haciendo complicado el avance en estas
% áreas al ser más rápido y eficiente directamente utilizar personal humano para
% resolver estos problemas. Más adelante veremos las distintas soluciones pasadas
% y actuales utilizadas en esta área.

% ¿Podemos entonces distinguir entre un algoritmo o proceso fruto de la
% inteligencia y otro que se base en pura fuerza bruta y velocidad de cálculo?
% ¿Son todos los avances en estas áreas igualmente importantes?

% Las preguntas anteriores son preguntas válidas y son parte de la base que ha
% motivado este trabajo.
% Es necesario mirar de forma crítica los avances que se están produciendo para
% comprobar la salud del área y, además, conseguir una perspectiva mejor de cara a
% futuros trabajos de investigación. Para esto es necesario entender el estado del
% arte y comparar los avances pasados con las bases de los avances actuales y
% hacer una comparación exhaustiva entre estos.

% Dada la amplitud del área y la profundidad de las distintas subareas dentro de
% la misma y contando con la limitación que lleva aparejada un trabajo de este
% tipo, no se profundizará por igual en las distintas áreas si no que se realizará
% un estudio somero del campo en general para luego profundizar en algunas áreas
% concretas, en las cuales se analizarán los avances pasados poniendo estos en
% relación al estado del arte de la materia con el ánimo de comparar la amplitud,
% alcance y estado del desarrollo actual.

% Las áreas sobre las cuales se pretende hacer especial énfasis son el aprendizaje
% profundo – Deep learning -, lingüística computacional o ingeniería lingüística,
% Razonamiento de sentido común y, por último, los sistemas de ayuda a la
% decisión.

% El trabajo está estructurado de forma que primero se presentará el contexto
% histórico de cada campo para luego, de forma ordenada, comparar ese contexto
% histórico con sus avances y desarrollo con el estado del arte desarrollado.

% Aunque las conclusiones que se extraigan estarán por fuerza limitadas al campo
% de alcance de este trabajo, al ser campos reconocidos e importantes dentro del
% área, no es descabellado extrapolar las conclusiones obtenidas al resto de
% campos no tratados.

% \section{Título del proyecto}

% En la portada ---y otras páginas de presentación--- el nombre o título del
% proyecto debe aparecer sin comillas, cursiva u otros formatos. Si se cita el
% título de otra obra, o el nombre de un capítulo sí debe aparecer entre
% comillas. Por cierto, las comillas que deben usarse en castellano son las
% «latinas», dejando las ``inglesas'' para los raros casos en los que aparezca una
% cita en el cuerpo otra~\cite{sousa}.


% \section{Estructura del documento}

% Pueden incluirse aquí una sección con algunos consejos para la lectura del
% documento dependiendo de la motivación o conocimientos del lector.  También
% puede ser útil incluir una lista con el nombre y finalidad de cada uno de los
% capítulos restantes.

% \begin{definitionlist}
%   \chapterTitle{objetivos}{Hola mundo}
%   \chapterTitle{antecedentes}{Hola mundo}
%   \chapterTitle{desarrollo}{Hola mundo}
%   \chapterTitle{resultados}{Hola mundo}
%   \chapterTitle{conclusiones}{Hola mundo}
%   \chapterTitle{lineas-futuras}{Hola mundo}
% \end{definitionlist}
