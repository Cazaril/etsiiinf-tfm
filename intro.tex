%&latex2
\chapter{Introducción}
\label{chap:introduccion}

\drop{E}{sto} se llama «letra capital» y debería utilizarse únicamente al
comienzo de capítulo como artificio decorativo. Para que resulte estéticamente
adecuada, este primer párrafo debería tener más del doble de líneas de lo que
ocupe verticalmente la letra capital (dos en este caso). El capítulo de
introducción debe dar al lector una perspectiva básica ---pero completa--- del
problema que se pretende abordar, pero también de la estrategia y enfoque que el
autor propone para su resolución. El lector debería poder determinar si este
documento le interesa leyendo únicamente la introducción.


\section{Título del proyecto}

En la portada ---y otras páginas de presentación--- el nombre o título del
proyecto debe aparecer sin comillas, cursiva u otros formatos. Si se cita el
título de otra obra, o el nombre de un capítulo sí debe aparecer entre
comillas. Por cierto, las comillas que deben usarse en castellano son las
«latinas», dejando las ``inglesas'' para los raros casos en los que aparezca una
cita en el cuerpo otra~\cite{sousa}.


\section{Estructura del documento}

Pueden incluirse aquí una sección con algunos consejos para la lectura del
documento dependiendo de la motivación o conocimientos del lector.  También
puede ser útil incluir una lista con el nombre y finalidad de cada uno de los
capítulos restantes.

\begin{definitionlist}
  \chapterTitle{objetivos}{Hola mundo}
  \chapterTitle{antecedentes}{Hola mundo}
  %\chapterTitle{desarrollo}{Hola mundo}
  %\chapterTitle{resultados}{Hola mundo}
  %\chapterTitle{conclusiones}{Hola mundo}
  %\chapterTitle{lineas-futuras}{Hola mundo}
\end{definitionlist}
