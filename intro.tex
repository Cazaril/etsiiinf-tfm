%&latex2
\chapter{Introducción}
\label{chap:introduccion}

\drop{A} lo largo de las décadas, se han visto las distintas olas de la
Inteligencia Artificial, sus consecuencias y los periodos intermedios conocidos
como inviernos. Existe un punto de similitud entre las distintas olas: Se
prometieron soluciones y avances que nunca llegaron y se menospreció la
dificultad computacional de los problemas. En el contexto actual en el cual se
ha vuelto a hablar de la Inteligencia Artificial, es interesante comprobar
exactamente la calidad de los avances del área y si se ha conseguido un avance
cualitativo con respecto a las técnicas de hace 30 años.

Siempre que se habla de la IA como área, existen dos grandes problemas:
\begin{itemize}
\item El efecto IA.Que resume la dificultad que existe cuando, al conseguir un
  proceso conocido para un problema, dicho proceso pasa a no ser considerado
  IA.Esto es, la misma comprensión de un algoritmo lo despoja de inteligencia
  ya que se elimina ese halo de subjetividad humana que rodea al problema.
\item Como resultado de la explosión de las distintas burbujas en los
  sucesivos inviernos de la IA, se ha establecido una desconfianza hacia todo
  lo relacionado con esta área y muchas de las salidas industriales han
  cambiado de nombre para librarse de ese estigma.
\end{itemize}
  
Estos problemas son solo la punta del iceberg, puesto que al ser un área
multidisciplinar y de investigación, aúna las partes negativas de áreas con
estas características – Necesidad de equipo con muy alta preparación, necesidad
de inversión fuerte, baja tasa de retorno de la inversión, etc – haciéndola un
objetivo de recortes inmediatos en tiempos de escasez económica.

Si podemos dividir el campo de la IA en sus tres pilares históricos fundamentales:
\begin{itemize}
\item Aprendizaje y razonamiento artificial.
\item Robótica.
\item Procesamiento del lenguaje natural.
\end{itemize}

Como es comprensible, la amplitud de un análisis que abarque por completo este
área resulta inviable tanto por extensidad como por alcance del presente
trabajo. Nos centraremos en el mismo en sólo dos de estos pilares: En el
Aprendizaje y razonamiento artificial y en el procesamiento del lenguaje natural
aunque es de rigor comentar la grandísima evolución ocurrida en el campo de la
robótica, aumentando las capacidades de acción y reacción de los distintos
componentes, dotándolos además de mejores sistemas motrices y mejor capacidad
propioceptiva.

Es interesante avanzar el problema de la explosión combinatorial del cual
hablaremos en profundidad más adelante. Los problemas a los que se suele
enfrentar la IA suelen ser problemas cuyo comportamiento responde a esta
explosión: El espacio de soluciones se hace cada vez más grande de forma
exponencial con respecto al tamaño de la muestra.

Este comportamiento hace difícil solucionar los problemas de forma clásica, es
decir, buscando la mejor solución generando todas las posibles soluciones del
problema. Es en este punto en el que entra la IA, proporcionando heurísticas –
más adelante se hablará de ellas – que permitan ahorrarse al algoritmo de
resolución la generación de soluciones no válidas. Dichas heurísticas suelen ser
guías más o menos ingeniosas encontradas por un ser humano pero también pueden
ser el fruto de un algoritmo de categorización automática, que de forma propia
ha creado dichas guías. A pesar de que la IA suele ser la herramienta con la
cual abordar este tipo de problemas, también ha sido necesario el desarrollo
técnico de los ordenadores – potencia de cálculo, cantidad de memoria disponible
– que permitiera a las distintas técnicas medirse con el problema a
resolver. Pero si ha sido necesario ese salto en la capacidad de cómputo, es
aceptable la duda de si dicha técnica es realmente una técnica válida.

Con esta duda en mente, se pretende realizar un análisis exhaustivo de cada área
a tratar, comentando los avances cualitativos o cuantitativos de cada
investigación para después hacer una comparación real con los avances más
recientes. Con esto se pretende hacer, además de un estado del arte, una
revisión crítica de los distintos avances y de su impacto real.

Antes de entrar en materia, se necesita comentar al menos de manera somera el
método de análisis y de comparación entre los distintos avances; Pongamos por
caso el desarrollo de la tecnología WiFi, cuya primer standar fue propuesto en
1997 como protocolo número 802.11. Obviamente, previo a dicho standar hubo otros
sistemas precursores -Como por ejemplo, WaveLAN en 1988- que hicieron posible la
creación de un standar que aunase las características deseables en dicha
tecnología; ¿Así pues, qué desarrollo o investigación tiene más relevancia? La
formalización del standar hizo posible muchos de los mecanismos que a día de hoy
nos parecen cotidianos pero, sin un camino previo, dicho standar no habría
llegado en el momento en que llegó: Es razonable pensar que el desarrollo
primero tiene más relevancia, pero no siempre es así; Sea como sea, en este
documento siempre se procurará hacer mención a ese trabajo primero que facilita
el camino, pero no se analizará el trabajo en función de su fecha de
investigación si no en su alcance academico-industrial: El trabajo más relevante
es aquel que consiga ser base de más trabajos o mejoras futuras, la
investigación más relevante es aquella que consigue hacer terreno fértil para
investigaciones posteriores sin entrar a considerar las aplicaciones técnicas de
la investigación en sí misma.
% \section{Estructura del documento}

% Pueden incluirse aquí una sección con algunos consejos para la lectura del
% documento dependiendo de la motivación o conocimientos del lector.  También
% puede ser útil incluir una lista con el nombre y finalidad de cada uno de los
% capítulos restantes.

% \begin{definitionlist}
%   \chapterTitle{objetivos}{Hola mundo}
%   \chapterTitle{antecedentes}{Hola mundo}
%   \chapterTitle{desarrollo}{Hola mundo}
%   \chapterTitle{resultados}{Hola mundo}
%   \chapterTitle{conclusiones}{Hola mundo}
%   \chapterTitle{lineas-futuras}{Hola mundo}
% \end{definitionlist}
